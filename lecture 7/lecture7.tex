\section{Lecture 7 Exercises}

\subsection{Dirichlet is not Neumann}

\textbf{Note from discussion forum}: in this exercise, we need $c=0$.

Let $(b, 0)$ be a graph over $(X,m)$ such that $m(X)=1$ and $\lambda_0 := \inf\sigma(L^{(D)})>0$. Show that $Q^{(D)}\ne Q^{(N)}$.

\paragraph{Solution:}
Note that by definition, $Q^{(N)}$ is an extension of $Q^{(D)}$ --- see the first two pages of Lecture 6. Assume for contradiction that $Q^{(D)}=Q^{(N)}$, i.e.\ in particular $D(Q^{(D)})=D(Q^{(N)})$. Recall the definition of the domain of the Neumann form, namely
\begin{equation*}
	D(Q^{(N)}) = \mathcal{D}\cap \ell^2(X,m),
\end{equation*}
where $\mathcal{D} := \{f\in C(X): \sum_{x\in X}b(x,y)[f(x)-f(y)]^2<\infty\}$ denotes the space of functions with finite energy. Since $m(X)=1$, we have that
\begin{equation*}
	1 = \sum_{x\in X}m(x) = \sum_{x\in X}1\cdot m(x) = \|\mathds{1}\|^2_2
\end{equation*}
where $\mathds{1}$ denotes the constant function with value $1$. Therefore $\mathds{1}\in \ell^2(X,m)$, and since it is clear that $\mathds{1}\in \mathcal{D}$, we find that $\mathds{1}\in D(Q^{(N)}) = D(Q^{(D)})$. However, since $L^{(D)}\mathds{1}=0$, this implies $\lambda_0 = \inf\sigma(L^{(D)})=0$, which is a contradiction,. Hence $Q^{(D)}\ne Q^{(N)}$ as required.

\subsection{Bounded functions in the domain form an algebra}
Let $(X,\mu)$ be a $\sigma$-finite measure space, and let $Q$ be a Dirichlet form on $L^2(X,\mu)$ with domain $D(Q)$. Show that $D(Q)\cap L^\infty(X,\mu)$ is an algebra.

\paragraph{Solution:}

Obviously $D(Q)\cap L^\infty(X,\mu)$ is a vector space, so it remains to show that it is closed under multiplication.

By Lemma 5.11, we have
\begin{equation*}
	\braket{(I-e^{-tL})(fg), fg} \le 2\|g\|^2_\infty \braket{(I-e^{-tL})f,f} + 2\|f\|^2_\infty \braket{(I-e^{-tL})g,g}
\end{equation*}
for all $f,g\in L^2(X,\mu)\cap L^\infty(X,\mu)$ and all $t>0$. Dividing both sides by $t$, we obtain
\begin{equation}
	Q^{(t)}(fg) \le 2\|g\|^2_\infty Q^{(t)}(f) + 2\|f\|^2_\infty Q^{(t)}(g).
\end{equation}
for all $t>0$, where $Q^{(t)}(f) := \frac{1}{t}\braket{(I-e^{-tL})f,f}$. Now assume that $f, g\in D(Q)\cap L^\infty(X,\mu) \subset L^2(X,\mu)\cap L^\infty(X,\mu)$. By Lemma 5.12, we may take $t\downarrow 0$ in the above inequality and deduce
\begin{equation*}
	Q'(fg) \le 2\|g\|^2_\infty Q'(f) + 2\|f\|^2_\infty Q'(g) = 2\|g\|^2_\infty Q(f) + 2\|f\|^2_\infty Q(g),.
\end{equation*}
where we have used that $f, g\in D(Q)$ in the final equality. Hence $Q'(fg)<\infty$, which implies (by Lemma 5.12 again) that $fg\in D(Q)$. Clearly $fg\in L^\infty(X,\mu)$ as well, and thus $fg \in D(Q)\cap L^\infty(X,\mu)$.