\section{Lecture 1 -- Exercises}

\subsection{Exercise 1 (Normal Contractions)}
\begin{itemize}
	\item [(a)]
		Show that the following maps from $\mathbb{R}$ to $\mathbb{R}$ are normal contractions:
		\begin{itemize}
			\item 
				$C_{+}:t\mapsto t\vee 0$
			\item 
				$C_{-}:t\mapsto (-t)\vee 0$
			\item 
				$C_{(-\infty,1]}:t\mapsto t\wedge 1$
			\item 
				$C_{[0,1]}:t\mapsto 0\vee(t\wedge 1)$
		\end{itemize}
		For which $a\leq b$ is $C_{[a,b]}:t\mapsto a\vee(t\wedge b)$ a normal contraction?

	\item [(b)]
		Let $X$ be a finite set and let $\mathcal{Q}$ be a symmetric bilinear form on $C(X)$. Show that $\mathcal{Q}$ is compatible with normal contractions if it is compatible with the map $C_{(-\infty,1]}$.
\end{itemize}

\paragraph{Solution:}
\begin{itemize}
	\item [(a)]
		\begin{itemize}
			\item 
				$C_{+}(0) = 0$.\\
				\begin{equation*}
					\begin{aligned}
						\left|C_{+}(s) - C_{+}(t)\right| &= \left|(s\vee 0) - (t\vee 0)\right|\\
						&=
						\begin{cases}
							0 &\text{if }s,t\leq 0\\
							|s - t| &\text{if }s,t>0\\
							|s| &\text{if }s\geq 0>t
						\end{cases} \leq |s - t|
					\end{aligned}
				\end{equation*}
			\item 
				$C_{-}(0) = 0$.\\
				\begin{equation*}
					\begin{aligned}
						|C_{-}(s) - C_{-}(t)| &= |(-s)\vee 0 - (-t)\vee 0|\\
						&=
						\begin{cases}
							|t - s| &\text{if }s,t\leq 0\\
							0 &\text{if }s,t>0\\
							|t| &\text{if }s\geq 0>t
						\end{cases} \leq |s - t|
					\end{aligned}
				\end{equation*}
			\item 
				$C_{(-\infty,1]}(0) = 0$.\\
				\begin{equation*}
					\begin{aligned}
						|C_{(\infty,1]}(s) - C_{(\infty,1]}(t)| &= |s\wedge 1 - t\wedge 1|\\
						&=
						\begin{cases}
							|s - t| &\text{if }s,t\leq 1\\
							0 &\text{if }s,t>1\\
							|s - 1| &\text{if }s\leq 1<t
						\end{cases} \leq |s - t|
					\end{aligned}
				\end{equation*}
			\item 
				$C_{[0,1]}(0) = 0$.\\
				\begin{equation*}
					\begin{aligned}
						|C_{[0,1]}(s) - C_{[0,1]}(t)| &= |0\vee(s\wedge 1) - 0\vee(t\wedge 1)|\\
						&=
						\begin{cases}
							|s - t| &\text{if }s,t\in[0,1]\\
							0 &\text{if }s,t\not\in[0,1]\\
							|s - 1| &\text{if }s\in[0,1]\text{ and }t\not\in[0,1]
						\end{cases} \leq |s - t|
					\end{aligned}
				\end{equation*}
		\end{itemize}
		To determine $a,b$ notice if $b<0$ then
		\begin{equation*}
			C_{[a,b]}(0) = a\vee(0\wedge b) = a\wedge b \neq 0
		\end{equation*}
		So we must have $b\geq 0$ in which case,
		\begin{equation*}
			C_{[a,b]}(0) = a\vee 0 = 0 \iff a\leq 0
		\end{equation*}
		Now to ensure the contraction property,
		\begin{equation*}
			\begin{aligned}
				|C_{[a,b]}(s) - C_{[a,b]}(t)| &= |a\vee(s\wedge b) - a\vee(t\wedge b)|\\
				&= 
				\begin{cases}
					|s - t| &\text{if }s,t\in[a,b]\\
					0 &\text{if }s,t\not\in[a,b]\\
					|s - b| &\text{if }s\in[a,b]\text{ and }t\not\in[a,b]
				\end{cases} \leq |s - t|
			\end{aligned}
		\end{equation*}
		Therefore if $a\leq 0\leq b$ then $C_{[a,b]}$ is a normal contraction.
\end{itemize}

\subsection{Exercise 2 (First Beurling-Deny criterion)}
Let $X$ be a finite set and let $\mathcal{Q}$ be a symmetric bilinear form over $X$. For any $f\in C(X)$, let $f_{+}=f\vee 0$ bet the \textit{positive part} and let $f_{-}=(-f)\vee 0$ be the \textit{negative part} of $f$.

Show the following equivalence:
\begin{itemize}
	\item [(i)]
		$\mathcal{Q}(|f|) \leq \mathcal{Q}(f)$ for all $f\in C(X)$.
	\item [(ii)]
		$\mathcal{Q}(f_{+},f_{-}) \leq 0$ for all $f\in C(X)$.
	\item [(iii)]
		$\mathcal{Q}(f\vee g) + \mathcal{Q}(f\wedge g) \leq \mathcal{Q}(f) + \mathcal{Q}(g)$ for all $f,g\in C(X)$.
\end{itemize}
and for $\mathcal{Q}$ positive show that this is also equivalent to:
\begin{itemize}
	\item [(iv)]
		$\mathcal{Q}(f_{+}) \leq \mathcal{Q}(f)$ for all $f\in C(X)$.
\end{itemize}

\paragraph{Solution:}

\paragraph{(i)$\implies$(ii)}
By properties of symmetric forms,
\begin{equation*}
	\begin{aligned}
		\mathcal{Q}(f_{=}) + 2\mathcal{Q}(f_{+},f_{-}) + \mathcal{Q}(f_{-}) &= \mathcal{Q}(|f|)\\
		&\leq \mathcal{Q}(f) = \mathcal{Q}(f_{+}) - 2\mathcal{Q}(f_{+},f_{-}) + \mathcal{Q}(f_{-})
	\end{aligned}
\end{equation*}
Then rearranging gives,
\begin{equation*}
	\mathcal{Q}(f_{+},f_{-}) \leq 0 \quad \forall f\in C(X)
\end{equation*}

\paragraph{(ii)$\implies$(i)}
By properties of symmetric forms,
\begin{equation*}
	\begin{aligned}
		\mathcal{Q}(f) &= \mathcal{Q}(f_{+}) - 2\mathcal{Q}(f_{+},f_{-}) + \mathcal{Q}(f_{-})\\
		&\geq \mathcal{Q}(f_{+}) + 2\mathcal{Q}(f_{+},f_{-}) + \mathcal{Q}(f_{-}) = \mathcal{Q}(|f|) \quad \forall f\in C(X)
	\end{aligned}
\end{equation*}

\paragraph{(i)$\implies$(iii)}
We can write $f\vee g = \frac{1}{2}(f+g)+\frac{1}{2}|f-g|$ and $f\wedge g = \frac{1}{2}(f+g)-\frac{1}{2}|f-g|$. Then,
\begin{equation*}
	\begin{aligned}
		\mathcal{Q}(f\vee g) &= \frac{1}{4}\mathcal{Q}(f+g) + \frac{1}{2}\mathcal{Q}(f+g,|f-g|) + \frac{1}{4}\mathcal{Q}(|f-g|)\\
		&\leq \frac{1}{4}\mathcal{Q}(f+g) + \frac{1}{2}\mathcal{Q}(f+g,|f-g|) + \frac{1}{4}\mathcal{Q}(f-g)\\
		&= \frac{1}{4}\mathcal{Q}(f) + \frac{1}{2}\mathcal{Q}(f,g) + \frac{1}{4}\mathcal{Q}(g) + \frac{1}{2}\mathcal{Q}(f+g,|f-g|) + \frac{1}{4}\mathcal{Q}(f) - \frac{1}{2}\mathcal{Q}(f,g) + \frac{1}{2}\mathcal{Q}(g)\\
		&= \frac{1}{2}\mathcal{Q}(f) + \frac{1}{2}\mathcal{Q}(g) + \frac{1}{2}\mathcal{Q}(f+g,|f-g|)
	\end{aligned}
\end{equation*}
By the same method,
\begin{equation*}
	\mathcal{Q}(f\wedge g) \leq \frac{1}{2}\mathcal{Q}(f) + \frac{1}{2}\mathcal{Q}(g) - \frac{1}{2}\mathcal{Q}(f+g,|f-g|)
\end{equation*}
and so,
\begin{equation*}
	\mathcal{Q}(f\vee g) + \mathcal{Q}(f\wedge g) \leq \mathcal{Q}(f) + \mathcal{Q}(g)
\end{equation*}

\paragraph{(iii)$\implies$(i)}
By the same calculations as above,
\begin{equation*}
	\frac{1}{2}\mathcal{Q}(f+g) + \frac{1}{2}\mathcal{Q}(|f-g|) \leq \mathcal{Q}(f) + \mathcal{Q}(g)
\end{equation*}
rearranging and expanding,
\begin{equation*}
	\begin{aligned}
		\frac{1}{2}\mathcal{Q}(|f-g|) &\leq \mathcal{Q}(f) + \mathcal{Q}(g) - \frac{1}{2}\mathcal{Q}(f) - \frac{1}{2}\mathcal{Q}(f,g) - \frac{1}{2}\mathcal{Q}(g)\\
		&= \frac{1}{2}\mathcal{Q}(f-g)
	\end{aligned}
\end{equation*}
Choosing $g=0$ gives the result.

\paragraph{(i)$\iff$(iv)}
We work with the assumption that $\mathcal{Q}$ is positive now. If (ii) holds then,
\begin{equation*}
	\mathcal{Q}(f^{+}) \leq \mathcal{Q}(f^{+}) - 2\mathcal{Q}(f^{+},f^{-}) + \mathcal{Q}(f^{-}) = \mathcal{Q}(f)
\end{equation*}
If (iv) holds then,
\begin{equation*}
	0 \leq \mathcal{Q}(f^{-}) - 2\mathcal{Q}(f^{+},f^{-})
\end{equation*}

\subsection{Exercise 3 (Harmonic functions and connected components)}
Let $(b,c)$ be a graph over a finite set measure space $(X,m)$ with associated Laplacian $L=L_{b,c,m}$ and let,
\begin{equation*}
	H = \left(f\in C(X)\colon Lf = 0\right)
\end{equation*}
be the subspace of harmonic functions. Show that $\dim H$ is equal to the number of connected components of $(b,c)$ on which $c$ vanished.

\paragraph{Solution:}
On any connected component where $c$ vanishes we have that any constant function (only on the connected component and zero everywhere else) is harmonic,
\begin{equation*}
	L\mathds{1}(x) = \frac{1}{m(x)}\sum_{y\in X}b(x,y)(\mathds{1}(x) - \mathds{1}(y)) = 0
\end{equation*}
Let $n\in\mathbb{N}$ denote the number of connected components of $(b,c)$. The constant functions $\mathds{1}$ on connected components where $c$ vanishes form a linearly independent set in $C(X)$ and so $V$. Hence $\dim H \geq n$.

If $f\in H$ let us consider it over a connected component where $c$ vanishes. Then,
\begin{equation*}
	(Lf)(x) = \frac{1}{m(x)}\sum_{y\in X} b(x,y)(f(x) - f(y)) = 0
\end{equation*}
for every $x$ in the connected component. In particular, if we take the cut-off of $f$ to any connected component where $c$ vanishes then we have a collection of linearly independent functions. The number of such functions would clearly equal the number of connected components where $c$ vanishes.

\subsection{Exercise 4 (Poisson equation for \texorpdfstring{$\alpha=0$)})}
Let $b$ be a graph over a finite set measure space $(X,m)$ (that is $c=0$) and let $L=L_{b,0,m}$ be the associated Laplacian. Furthermore let,
\begin{equation*}
	V:=\left\{f\in C(X)\colon \sum_{x\in X}f(x)m(x) = 0\right\}
\end{equation*}
Show that for each $f\in V$, there is a unique functions $u\in V$ such that,
\begin{equation*}
	Lu = f
\end{equation*}

\textit{Hint:} Observe that for the scalar product in $\ell^{2}(X,m)$, we have for all $f\in\ell^{2}(X,m)$,
\begin{equation*}
	\sum_{x\in X}f(x)m(x) = \braket{f,1}
\end{equation*}
