\section{Lecture 6 -- Exercises}

\subsection{Exercise 1 (Density of \texorpdfstring{$C_{c}$}{})}
Let $(X,m)$ be an infinite discrete measure space and $p\in[1,\infty]$. Show that $C_{c}(X)$ is dense in $\ell^{p}(X,m)$ if and only if $p\in[1,\infty)$.

\paragraph{Solution:}
We first recall the supremum norm definition,
\begin{equation*}
	\|f\|_{\infty}:=\inf\left\{c\geq 0 \colon \sum_{|f(x)|>c}m(x)=0\right\}
\end{equation*}

($\implies$): Suppose that $\overline{C_{c}(X)}^{\|\cdot\|_{\infty}}=\ell^{\infty}(X,m)$. Since $\mathds{1}\in\ell^{\infty}(X,m)$ then there exists $(f_{n})\in C_{c}(X)$ such that $\|f_{n}-\mathds{1}\|_{\infty}\to 0$. By definition, $f_{n}$ has finite support for each $n\in\mathbb{N}$ and so for every $f_{n}$ there exists $x_{n}\in X$ such that $f(x_{n})=0$ and hence $\|f_{n}-\mathds{1}\|_{\infty}=1$ for all $n$. This is clearly a contradiction and so we must have $p\in[1,\infty)$.

($\impliedby$): For any $f\in\ell^{p}(X,m)$ we can assume $\text{supp}(f)$ is countable. Setting $K_{n}=\left\{x_{1},\ldots,x_{n}\colon x_{i}\in \text{supp}(f),\, i=1,\ldots,n\right\}$ then $K_{n}$ is finite, $K_{n}\subset K_{n+1}$ and $\bigcup_{n}K_{n}=\text{supp}(f)$. Taking the sequence $(\mathds{1}_{K_{n}}f)$ it is clear that $\mathds{1}_{K_{n}}f\to f$ pointwise and $|\mathds{1}_{K_{n}}f|\leq f$ a.e, hence by dominated convergence $\|\mathds{1}_{K_{n}}f - f\|_{p}\to 0$ for $p\in[1,\infty)$. So $\overline{C_{c}(X)}^{\|\cdot\|_{p}}=\ell^{p}(X,m)$ for $p\in[1,\infty)$.

\subsection{Exercise 2 (Inclusion of \texorpdfstring{$\ell^{p}$}{} spaces)}
\emph{Note}: we modify the question so that we will solve the bonus exercise as well.

Let $(X,m)$ be a discrete measure space.
\begin{enumerate}[(a)]
	\item 
		For any $p\in [1,\infty)$, show the equivalence of the following statements:
		\begin{enumerate}[(i)]
			\item 
				$\ell^{p}(X,m)\subseteq\ell^{\infty}(X,m)$.
			\item 
				$\ell^{p}(X,m)\subseteq C_{0}(X):=\overline{C_{c}(X)}^{\|\cdot\|_{\infty}}$.
			\item 
				There exists $\alpha>0$ such that $m\geq\alpha$.
		\end{enumerate}
	\item 
		For any $p\in [1, \infty)$, show the equivalence of the following statements:
		\begin{enumerate}[(i)]
			\item 
				$\ell^{p}(X,m)\supseteq\ell^{\infty}(X,m)$.
			\item 
				$m(X)<\infty$.
		\end{enumerate}
\end{enumerate}

\paragraph{Solution:}

\begin{enumerate}[(a)]
	\item 
		(i)$\iff$(ii): The inclusion in (i) is equivalent to the existence of $c>0$ such that $\|f\|_{\infty}\leq c\|f\|_{p}$ for all $f\in\ell^{p}(X,m)$. From Exercise 1 we have $\overline{C_{c}(X)}^{\|\cdot\|_{p}}=\ell^{p}(X,m)$, so it is sufficient to show $\overline{C_{c}(X)}^{\|\cdot\|_{p}}\subseteq \overline{C_{c}(X)}^{\|\cdot\|_{\infty}}$. Let $(f_{n})\in C_{c}(X)$ be a Cauchy sequence with respect to $\|\cdot\|_{p}$. Then there exists $f\in\overline{C_{c}(X)}^{\|\cdot\|_{p}}$ such that $\|f_{n}-f\|_{p}\to 0$. By the earlier inequality we also have that $(f_{n})$ is Cauchy with respect to $\|\cdot\|_{\infty}$, $f\in\ell^{\infty}(X,m)$ and $\|f_{n}-f\|_{\infty}\to 0$. So $f\in\overline{C_{c}(X)}^{\|\cdot\|_{\infty}}$. Therefore
		\begin{equation*}
			\ell^p(X,m)=\overline{C_{c}(X)}^{\|\cdot\|_{p}}\subseteq\overline{C_{c}(X)}^{\|\cdot\|_{\infty}}=C_0(X).
		\end{equation*}
		The reverse implication is trivial since $C_{c}(X)\subset\ell^{\infty}(X,m)$.

		(i)$\iff$(iii): If (i) is satisfied then from above we have,
		\begin{equation*}
			\frac{1}{c}\|f\|_{\infty}\leq \|f\|_{p} \quad \forall f\in\ell^{p}(X,m).
		\end{equation*}
		For any $x\in X$, by taking $f=\mathds{1}_{x}$ we obtain
		\begin{equation*}
			\frac{1}{c}\leq m(x)^{1/p}.
		\end{equation*}
		Setting $\alpha=\frac{1}{c^p}$ gives (iii). Now if (iii) is satisfied, then for any $f\in\ell^{p}(X,m)$ we have
		\begin{equation*}
			\alpha |f(x)|^p \leq |f(x)|^p m(x) \leq \|f\|^p_{p}.
		\end{equation*}
		Therefore
		\begin{equation*}
			\sup_{x\in X}|f(x)| \le \frac{1}{\alpha^{1/p}}\|f\|_p
		\end{equation*}
		which implies (i).
	\item 
		First suppose $\ell^{p}(X,m)\supseteq\ell^{\infty}(X,m)$. Since $\mathds{1}\in\ell^{\infty}(X,m)$ we have,
		\begin{equation*}
			m(X)=\sum_{x\in X}m(x) = \|\mathds{1}\|^p_{p} < \infty.
		\end{equation*}
		Conversely if $m(X)<\infty$ then for any $f\in\ell^{\infty}(X,m)$ we have,
		\begin{equation*}
			\|f\|^p_p = \sum_{x\in X}|f(x)|^p m(x) \leq \|f\|^p_{\infty}\sum_{x\in X}m(x) = \|f\|^p_{\infty}m(X) < \infty.
		\end{equation*}
		Hence, $\ell^{p}(X,m)\supseteq\ell^{\infty}(X,m)$.
\end{enumerate}

\subsection{Exercise 3 (Boundedness)}
Let $(b,c)$ be a graph over $(X,m)$. Show that $\mathcal{L}$ is bounded on $\ell^2(X,m)$ if and only if it is bounded on $\ell^p(X,m)$ for some $p\in [1,\infty]$.

\paragraph{Solution:}

We recall the following facts from Lecture 3: the \emph{weighted degree function} of a graph is defined by
\begin{equation*}
	\mathrm{Deg}(x):= \frac{1}{m(x)}\left[\sum_{y\in X}b(x,y) +c(x) \right],
\end{equation*}
and the Laplacian $\mathcal{L}$ is bounded on $\ell^2(X,m)$ if and only if $\mathrm{Deg}(\cdot)$ is bounded on $X$ (see Theorem 2.18). Thus it suffices to prove that if $\mathcal{L}$ is bounded on $\ell^p(X,m)$ for some $p\in [1,\infty]$, then the weighted degree function is bounded on $X$.

We write $\|\cdot\|_p$ for the norm on $\ell^p(X,m)$. Let $\kappa_p$ be the operator norm of $\mathcal{L}$ on $\ell^p(X,m)$, i.e.\ $\kappa_p = \sup_{\|f\|_p \le 1}\|\mathcal{L}f\|_p$. We now compute
\begin{equation}
	\mathcal{L}\mathds{1}_x(y) = \begin{cases}
		\mathrm{Deg}(x) \qquad y=x \\ -\frac{b(x,y)}{m(y)} \qquad y\ne x.
		\end{cases}
\end{equation}
The above calculation shows that $\mathcal{L}\mathds{1}_x(y)\le 0$ if $y\ne x$. We treat the case $p=\infty$ first. Fix an arbitrary $x\in X$, and observe that
\begin{equation*}
	\|\mathcal{L}\mathds{1}_x\|_\infty = \sup_{y\in X}|\mathcal{L}\mathds{1}_x(y)| = \mathcal{L}\mathds{1}_x(x) = \mathrm{Deg}(x).
\end{equation*}
Clearly $\|\mathds{1}_x\|_\infty =1$ for all $x\in X$, so it follows that
\begin{equation*}
	\sup_{x\in X}\mathrm{Deg}(x) = \sup_{x\in X}\|\mathcal{L}\mathds{1}_x\|_\infty \le \kappa_\infty \sup_{x\in X}\|\mathds{1}_x\|_\infty = \kappa_\infty.
\end{equation*}
Hence $\mathrm{Deg}(\cdot)$ is bounded on $X$.

We have that $\|\mathds{1}_x\|_p=m(x)^{1/p}$ for each $x\in X$, where $1/\infty := 0$. Thus if $1\le p<\infty$, it follows that
\begin{equation*}
	\sum_{y\in X}b(x,y) + c(x) = m(x)\mathcal{L}\mathds{1}_x(x) = \braket{\mathds{1}_x, \mathcal{L}\mathds{1}_x} \le \|\mathcal{L}\mathds{1}_x\|_p \|\mathds{1}\|_{p'},
\end{equation*}
where $\braket{\cdot, \cdot}$ is the duality pairing between $\ell^p(X,m)$ and $\ell^{p'}(X,m)$ given by
\begin{equation*}
	\braket{f,g} = \sum_{x\in X}f(x)g(x)m(x), \qquad f\in \ell^{p'}(X,m), g\in \ell^p(X,m).
\end{equation*}
Therefore
\begin{equation}
	m(x)\mathrm{Deg}(x) \le \|\mathcal{L}\mathds{1}_x\|_p \|\mathds{1}\|_{p'} \le \kappa_p m(x)^{1/p}m(x)^{1/p'} = \kappa_p m(x),
\end{equation}
which yields the bound $\sup_{x\in X}\mathrm{Deg}(x) \le \kappa_p$.

\subsection{Exercise 4: Forms in between Dirichlet and Neumann}
Let $(b,c)$ be a graph over $(X,m)$ and $U\subseteq X$. Define
\begin{equation*}
\begin{aligned}
	D(Q^{(U)}) &= \overline{\{u\in D(Q^{(N)}) : U \cap \supp u \text{ is finite} \}}^{\|\cdot\|_{Q^{(N)}}} \\
	Q^{(U)}(f,g) &= Q^{(N)}(f,g).
\end{aligned}		
\end{equation*}
Show that
\begin{enumerate}[(a)]
	\item $Q^{(U)}$ is a Dirichlet form;
	\item $Q^{(D)}\subseteq Q^{(U)} \subseteq Q^{(N)}$. Furthermore, show that $Q^{(X\setminus F)} = Q^{(D)}$ and $Q^{(F)}\subseteq Q^{(N)}$ for any finite subset $F\subseteq X$.
\end{enumerate}

\paragraph{Solution:}

We define
\begin{equation}
	\mathcal{E}_U := \{u\in D(Q^{(N)}) : U \cap \supp u \text{ is finite} \}.
\end{equation}
for every subset $U\subseteq X$.

(a): It is clear that $Q^{(U)}$ is a positive, symmetric form that is compatible with normal contractions, since these properties hold for $Q^{(N)}$. It remains to check that $Q^{(U)}$ is closed. For any $u, v\in D(Q^{(N)})$ and scalar $\lambda\in\mathbb{RR}$, note that $\supp (\lambda u)=\supp u$ and $\supp(u+v) \subseteq \supp u \cup \supp v$. Hence, if $u,v\in\mathcal{E}_U$, it follows that $\supp(\lambda u)\cap U$ and $\supp(u+v)\cap U$ are finite subsets as well, which shows that $\mathcal{E}_U$ is a vector space. Consequently, $D(Q^{(U)})$ is a closed subspace of the Banach space $(D(Q^{(N)}), \|\cdot\|_{Q^{(N)}})$, and thus is itself a Banach space with respect to the norm $\|\cdot\|_{Q^{(N)}}$. This proves that $Q^{(U)}$ is a closed form. Combining with the previous observations, we conclude that $Q^{(U)}$ is a Dirichlet form.

(b): It is immediate from the definition that $Q^{(U)}\subseteq Q^{(N)}$, i.e.\ $Q^{(U)}$ is a restriction of $Q^{(N)}$. On the other hand, if $u \in C_c(X)$, then $U\cap\supp u$ is a finite set for any subset $U\subseteq X$. Hence $C_c(X)\subseteq\mathcal{E}$. Consequently
\begin{equation*}
	D(Q^{(D)}) = \overline{C_c(X)}^{\|\cdot\|_{Q^{(N)}}} \subseteq \overline{\mathcal{E}}^{\|\cdot\|_{Q^{(N)}}} = D(Q^{(U)}),
\end{equation*}
hence $Q^{(D)}$ is a restriction of $Q^{(U)}$.

Let us now make some general observations.
\begin{enumerate}[(i)]
	\item If $U\subseteq V\subseteq X$, then $\mathcal{E}_V\subseteq \mathcal{E}_U$.
	\item If $U=X$, then clearly $\mathcal{E}_U = C_c(X)$. Moreover, if $U$ is a finite (possibly empty) subset of $X$, then $\mathcal{E}_U = D(Q^{(N)})$, since $U\cap\supp u$ is a finite subset for all $u\in D(Q^{(N)}).$.
\end{enumerate}
For a given finite subset $F\subseteq X$, assertion (ii) above shows that $Q^{(F)}=Q^{(N)}$. Now suppose $u\in \mathcal{E}_{X\setminus F}$. By definition, $(X\setminus F)\cap\supp u$ is a finite subset. However, since $F$ is finite, so is $F\cap\supp u$, and therefore $\supp u$ is finite. This shows that $\mathcal{E}_{X\setminus F}\subseteq C_c(X)$. Upon taking closures in the $Q^{(N)}$-norm, we conclude that $D(Q^{X\setminus F})=D(Q^{(D)})$ and thus $Q^{(X\setminus F)}=Q^{(D)}$.